\hypertarget{grassout_8C}{
\section{grassout.C File Reference}
\label{grassout_8C}\index{grassout.C@{grassout.C}}
}
{\tt \#include \char`\"{}../header/hpfem.h\char`\"{}}\par
\subsection*{Functions}
\begin{CompactItemize}
\item 
void \hyperlink{grassout_8C_a0}{grass\_\-sites\_\-header\_\-output} (\hyperlink{structTimeProps}{Time\-Props} $\ast$timeprops)
\begin{CompactList}\small\item\em this function writes the header for grass sites style output, the grass sites output format is correct and works, but importing the data into GIS packages such as ARCGIS is non trivial until you know the trick to it. Keith does not know the trick, I just wrote this in the format Alex Sorokine specified. \item\end{CompactList}\item 
void \hyperlink{grassout_8C_a1}{grass\_\-sites\_\-proc\_\-output} (\hyperlink{classHashTable}{Hash\-Table} $\ast$HT\_\-Elem\_\-Ptr, \hyperlink{classHashTable}{Hash\-Table} $\ast$HT\_\-Node\_\-Ptr, int myid, \hyperlink{structMatProps}{Mat\-Props} $\ast$matprops, \hyperlink{structTimeProps}{Time\-Props} $\ast$timeprops)
\begin{CompactList}\small\item\em this function writes one processors grass sites style output, the grass sites output format is correct and works, but importing the data into GIS packages such as ARCGIS is non trivial until you know the trick to it. Keith does not know the trick, I (Keith) just wrote this in the format Alex Sorokine specified. \item\end{CompactList}\end{CompactItemize}


\subsection{Function Documentation}
\hypertarget{grassout_8C_a0}{
\index{grassout.C@{grassout.C}!grass_sites_header_output@{grass\_\-sites\_\-header\_\-output}}
\index{grass_sites_header_output@{grass\_\-sites\_\-header\_\-output}!grassout.C@{grassout.C}}
\subsubsection[grass\_\-sites\_\-header\_\-output]{\setlength{\rightskip}{0pt plus 5cm}void grass\_\-sites\_\-header\_\-output (\hyperlink{structTimeProps}{Time\-Props} $\ast$ {\em timeprops})}}
\label{grassout_8C_a0}


this function writes the header for grass sites style output, the grass sites output format is correct and works, but importing the data into GIS packages such as ARCGIS is non trivial until you know the trick to it. Keith does not know the trick, I just wrote this in the format Alex Sorokine specified. 

\hypertarget{grassout_8C_a1}{
\index{grassout.C@{grassout.C}!grass_sites_proc_output@{grass\_\-sites\_\-proc\_\-output}}
\index{grass_sites_proc_output@{grass\_\-sites\_\-proc\_\-output}!grassout.C@{grassout.C}}
\subsubsection[grass\_\-sites\_\-proc\_\-output]{\setlength{\rightskip}{0pt plus 5cm}void grass\_\-sites\_\-proc\_\-output (\hyperlink{classHashTable}{Hash\-Table} $\ast$ {\em HT\_\-Elem\_\-Ptr}, \hyperlink{classHashTable}{Hash\-Table} $\ast$ {\em HT\_\-Node\_\-Ptr}, int {\em myid}, \hyperlink{structMatProps}{Mat\-Props} $\ast$ {\em matprops}, \hyperlink{structTimeProps}{Time\-Props} $\ast$ {\em timeprops})}}
\label{grassout_8C_a1}


this function writes one processors grass sites style output, the grass sites output format is correct and works, but importing the data into GIS packages such as ARCGIS is non trivial until you know the trick to it. Keith does not know the trick, I (Keith) just wrote this in the format Alex Sorokine specified. 

